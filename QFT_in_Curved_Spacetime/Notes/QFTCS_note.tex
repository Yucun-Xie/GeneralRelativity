\documentclass[12pt]{article}
%\usepackage[english]{babel}
\usepackage{graphicx}
%\usepackage{framed}
%\usepackage[normalem]{ulem}
\usepackage{indentfirst}
\usepackage{amsmath,amsthm,amssymb,amsfonts}
\numberwithin{equation}{subsection}
\usepackage[italicdiff]{physics}
\usepackage[T1]{fontenc}
\usepackage{wrapfig}
\usepackage{lmodern,mathrsfs}
\usepackage[inline,shortlabels]{enumitem}
\setlist{topsep=2pt,itemsep=2pt,parsep=0pt,partopsep=0pt}
\usepackage[dvipsnames]{xcolor}
\usepackage[utf8]{inputenc}
%\usepackage[letterpaper, top=0.5in,bottom=0.2in, left=0.5in, right=0.5in, footskip=0.3in, includefoot]{geometry}
\usepackage[letterpaper,left=0.5in, right=0.5in,top=0.7in,bottom=1in,footskip=0.3in,includefoot]{geometry}
\usepackage[most]{tcolorbox}
\usepackage{tikz,tikz-3dplot,tikz-cd,tkz-tab,tkz-euclide,pgf,pgfplots}
\pgfplotsset{compat=newest}
\usepackage{multicol}
\usepackage[bottom,multiple]{footmisc} 
%\usepackage[backend=bibtex,style=numeric]{biblatex}
%\addbibresource{bibliography}
\usepackage{hyperref}
\usepackage[nameinlink]{cleveref} 
\usepackage[titletoc, toc, page]{appendix}





\newtheoremstyle{mystyle}{}{}{}{}{\sffamily\bfseries}{.\\}{ }{}
\newtheoremstyle{cstyle}{}{}{}{}{\sffamily\bfseries}{}{ }{\thmnote{#3}}

\theoremstyle{mystyle}{\newtheorem{definition}{Definition}[subsection]}
%\theoremstyle{mystyle}{\newtheorem{proposition}[definition]{Proposition}}
\theoremstyle{mystyle}{\newtheorem{theorem}[definition]{Theorem}}
%\theoremstyle{mystyle}{\newtheorem{lemma}[definition]{Lemma}}
%\theoremstyle{mystyle}{\newtheorem{corollary}[definition]{Corollary}}
\theoremstyle{mystyle}{\newtheorem*{remark}{Remark}}
%\theoremstyle{mystyle}{\newtheorem*{remarks}{Remarks}}
\theoremstyle{mystyle}{\newtheorem{example}{Example}[subsection]}
\theoremstyle{mystyle}{\newtheorem{examples}{Examples}[subsection]}
%\theoremstyle{definition}{\newtheorem*{exercise}{Exercise}}
\theoremstyle{mystyle}{\newtheorem{cthm}{}[subsection]}


\tcolorboxenvironment{definition}{boxrule=0pt,boxsep=0pt,colback={red!10},left=8pt,right=8pt,enhanced jigsaw, borderline west={2pt}{0pt}{red},sharp corners,before skip=10pt,after skip=10pt,breakable}
%\tcolorboxenvironment{proposition}{boxrule=0pt,boxsep=0pt,colback={Orange!10},left=8pt,right=8pt,enhanced jigsaw, borderline west={2pt}{0pt}{Orange},sharp corners,before skip=10pt,after skip=10pt,breakable}
\tcolorboxenvironment{theorem}{boxrule=0pt,boxsep=0pt,colback={blue!10},left=8pt,right=8pt,enhanced jigsaw, borderline west={2pt}{0pt}{blue},sharp corners,before skip=10pt,after skip=10pt,breakable}
\tcolorboxenvironment{example}{boxrule=0pt,boxsep=0pt,colback={Green!10},left=8pt,right=8pt,enhanced jigsaw, borderline west={2pt}{0pt}{Green},sharp corners,before skip=10pt,after skip=10pt,breakable}
\tcolorboxenvironment{examples}{boxrule=0pt,boxsep=0pt,colback={violet!10},left=8pt,right=8pt,enhanced jigsaw, borderline west={2pt}{0pt}{violet},sharp corners,before skip=10pt,after skip=10pt,breakable}
\tcolorboxenvironment{proof}{boxrule=0pt,boxsep=0pt,blanker,borderline west={2pt}{0pt}{CadetBlue!80!white},left=8pt,right=8pt,sharp corners,before skip=10pt,after skip=10pt,breakable}
\tcolorboxenvironment{remark}{boxrule=0pt,boxsep=0pt,blanker,borderline west={2pt}{0pt}{Cyan},left=8pt,right=8pt,before skip=10pt,after skip=10pt,breakable}
%\tcolorboxenvironment{remarks}{boxrule=0pt,boxsep=0pt,blanker,borderline west={2pt}{0pt}{Green},left=8pt,right=8pt,before skip=10pt,after skip=10pt,breakable}
%\tcolorboxenvironment{example}{boxrule=0pt,boxsep=0pt,blanker,borderline west={2pt}{0pt}{Black},left=8pt,right=8pt,sharp corners,before skip=10pt,after skip=10pt,breakable}
%\tcolorboxenvironment{examples}{boxrule=0pt,boxsep=0pt,blanker,borderline west={2pt}{0pt}{Black},left=8pt,right=8pt,sharp corners,before skip=10pt,after skip=10pt,breakable}
\tcolorboxenvironment{cthm}{boxrule=0pt,boxsep=0pt,colback={orange!10},left=8pt,right=8pt,enhanced jigsaw, borderline west={2pt}{0pt}{orange},sharp corners,before skip=10pt,after skip=10pt,breakable}
\tcolorboxenvironment{equation}{boxrule=0pt,boxsep=0pt,colback={Magenta!10},left=8pt,right=8pt,enhanced jigsaw, borderline west={2pt}{0pt}{Magenta},sharp corners,before skip=10pt,after skip=10pt,breakable}
\tcolorboxenvironment{align}{boxrule=0pt,boxsep=0pt,colback={MidnightBlue!10},left=8pt,right=8pt,enhanced jigsaw, borderline west={2pt}{0pt}{MidnightBlue},sharp corners,before skip=10pt,after skip=10pt,breakable}



\usepackage[explicit]{titlesec}
\titleformat{\section}{\fontsize{24}{30}\sffamily\bfseries}{\thesection}{20pt}{#1}
\titleformat{\subsection}{\fontsize{16}{18}\sffamily\bfseries}{\thesubsection}{12pt}{#1}
\titleformat{\subsubsection}{\fontsize{10}{12}\sffamily\large\bfseries}{\thesubsubsection}{8pt}{#1}

\titlespacing*{\section}{0pt}{5pt}{5pt}
\titlespacing*{\subsection}{0pt}{5pt}{5pt}
\titlespacing*{\subsubsection}{0pt}{5pt}{5pt}

\newcommand{\sectionbreak}{\clearpage} %Start every section on a new page
\newcommand{\tbf}[1]{\textbf{#1}}
\newcommand{\p}{\partial}
\newcommand{\id}{\mathrm{d}}
%\newcommand{\Disp}{\displaystyle}
%\newcommand{\qe}{\hfill\(\bigtriangledown\)}
%\DeclareMathAlphabet\mathbfcal{OMS}{cmsy}{b}{n}
%\setlength{\parindent}{0.2in}
%\setlength{\parskip}{0pt}
%\setlength{\columnseprule}{0pt}

\title{\huge\sffamily\bfseries Quantum Field Theory in/of Curved Spacetime}
\author{\Large\sffamily Yucun Xie}
\date{\sffamily \today}

\begin{document}

\setlength{\abovedisplayskip}{3pt}
\setlength{\belowdisplayskip}{3pt}
\setlength{\abovedisplayshortskip}{0pt}
\setlength{\belowdisplayshortskip}{0pt}
\maketitle

%Custom colors for different environments
\definecolor{contcol1}{HTML}{72E094}
\definecolor{contcol2}{HTML}{24E2D6}
\definecolor{convcol1}{HTML}{C0392B}
\definecolor{convcol2}{HTML}{8E44AD}

\begin{tcolorbox}[
    title=Contents, fonttitle=\huge\sffamily\bfseries\selectfont,
    interior style={left color=contcol1!40!white,right color=contcol2!40!white},
    frame style={left color=contcol1!80!white,right color=contcol2!80!white},
    coltitle=black,top=2mm,bottom=2mm,left=2mm,right=2mm,drop fuzzy shadow,enhanced,breakable]
  \makeatletter
  \@starttoc{toc}
  \makeatother
\end{tcolorbox}

\newpage










\begin{tcolorbox}[
    title=Conventions, fonttitle=\large\sffamily\bfseries\selectfont,
    interior style={left color=convcol1!40!white,right color=convcol2!40!white},
    frame style={left color=convcol1!80!white,right color=convcol2!80!white},
    coltitle=black,top=2mm,bottom=2mm,left=2mm,right=2mm,drop fuzzy shadow,enhanced,breakable]
  \begin{enumerate}

    \item Greek index (e.g. $\alpha, \beta, \mu, \nu$) take value from \{0, 1, 2, 3\}.
          %\item Events denoted by cursive capitals  (e.g. $\mathscr{A}, \mathscr{B}, \mathscr{E}$).
    \item $(x^0, x^1, x^2, x^3) \equiv (t, x, y, z) \equiv x^{\alpha}$.
    \item Latin index (e.g.$ i, j, k$) take value from \{1, 2, 3\}.
    \item Natural units ($c=1$).
    \item Einstein summation convention. \[ds^2 = g_{\mu \nu} dx^{\mu} dx^{\nu}=
            \sum_{\mu=0}^{3} \sum_{\nu=0}^{3}g_{\mu \nu} dx^{\mu} dx^{\nu}\]
    \item Metric signature $(+, -, -, -)$.

  \end{enumerate}
\end{tcolorbox}

\newpage





























%begin here ---------------------------------------------------------------------------------------------
\section{QFT in Flat Spacetime Revisit}
In this chapter, we will review the quantum field theory in flat spacetime and carefully distinguish
the concepts only valid in flat spacetime and the concepts that could be generalized to curved spacetime.
\subsection{Scalar Field Construction}

Consider a massive scalar field \(\phi(t,x^{i})\) defined in spacetime point \((t,x^{i})\) satisfying the \textbf{Klein-Gordon equation}:
\begin{equation}
  (\Box + m^2) \phi = 0
\end{equation}

\begin{remark}
  Here we use the metric sign convention \((+,-,-,-)\); if we used other sign convention, the Klein-Gordon equation
  would read \((\Box - m^2) \phi = 0\).
\end{remark}

where the d'Alembertian \(\Box\) is defined as \(\Box = g^{\mu\nu}\p_{\mu}\p_{\nu}\).
In this chapter, we assume the spacetime is flat, so \(g^{\mu\nu}=\eta^{\mu\nu}\).\par
This equation could be obtained from the Lagrangian density
\[\mathcal{L} = \frac{1}{2}(\eta^{\mu\nu} \phi_{,\mu} \phi_{,\nu}- m^2 \phi^2)\]
by demanding the variations of the action \[S = \int \mathcal{L}(x) \mathrm{d}^{n}x\] vanish.
\par
The solution of the Klein-Gordon equation satisfying
\[f_{\mathbf{k}}(t,\mathbf{x})= Ae^{i(\mathbf{k}\cdot \mathbf{x}-\omega t)}\]

The dispersion relation is \[\omega = \sqrt{(\mathbf{|k|}^2 + m^2 )}.\]
We can rewrite the above mode functions use four-wave vector \(k^{\mu}=(\omega,\mathbf{k})\):
\begin{equation}\label{112}
  f_{\mathbf{k}}(x^{\mu})= Ae^{-i(k_\mu x^\mu)}
\end{equation}
The above solution is very similar to the solution of harmonic oscillators. However, there is a significant difference:
A harmonic oscillator only has one independent solution because it has a fixed, unique frequency.
This feature no longer holds for fields theory because we have an infinite number solution for each value of \(k\).
Therefore, we should construct a general solution by constructing a complete, orthonormal set of modes that any solution
can express as a linear combination of modes.
To achieve this, first define the inner product of mode functions:
\begin{definition}[Klein-Gordon inner product]
  \[(\phi_1,\phi_2)=-i\int_{\Sigma_t}\id^{n-1}x\;\left[\phi_1\p_t\phi^*_2-\phi_2^*\p_t\phi_1\right]\]
  Which is integral over constant-time hypersurface \(\Sigma_t\).
\end{definition}
From generalized Stoke's theorem:\[\int_{M}\id\omega=\int_{\p M}\omega\]
\hfill \break
it is easy to check that the inner product is independent of choose of the hypersurface.
By explicitly calculating the inner product:
\begin{align}
  \left(f_{\mathbf{k}},f_{\mathbf{k}'}\right)\propto&\left(e^{-ik^{\mu}x_{\mu}},e^{-ik'^{\nu}x_{\nu}}\right)\\
  =&-i\int_{\Sigma_t}\id^{n-1}x\;\left[e^{-i\omega t+\mathbf{k}\cdot\mathbf{x}}
  \p_{t}e^{i\omega' t-\mathbf{k'}\cdot\mathbf{x}}-e^{i\omega' t-\mathbf{k'}\cdot\mathbf{x}}
  \p_{t}e^{-i\omega t+\mathbf{k}\cdot\mathbf{x}}\right]\\
  =&-i\int_{\Sigma_t}\id^{n-1}x\;\left[e^{-i\omega t} e^{\mathbf{k}\cdot\mathbf{x}}
  \p_{t}e^{i\omega' t}e^{-\mathbf{k'}\cdot\mathbf{x}}-e^{i\omega' t}e^{-\mathbf{k'}\cdot\mathbf{x}}
  \p_{t}e^{-i\omega t}e^{\mathbf{k}\cdot\mathbf{x}}\right]\\
  =&\int_{\Sigma_t}\id^{n-1}x\;\left[e^{-i\omega t} e^{\mathbf{k}\cdot\mathbf{x}}
  \omega' e^{i\omega' t}e^{-\mathbf{k'}\cdot\mathbf{x}}+e^{i\omega' t}e^{-\mathbf{k'}\cdot\mathbf{x}}
  \omega e^{-i\omega t}e^{\mathbf{k}\cdot\mathbf{x}}\right]\\
  =&e^{-i\omega t}e^{i\omega' t}(\omega'+\omega)\int_{\Sigma_t}\id^{n-1}x\;
  \left[ e^{\mathbf{k}\cdot\mathbf{x}}e^{-\mathbf{k'}\cdot\mathbf{x}}\right]\\
  =&e^{i(\omega'-\omega)t}(\omega'+\omega)\int_{\Sigma_t}\id^{n-1}x\;
  \left[ e^{(\mathbf{k}-\mathbf{k'})\cdot\mathbf{x}}\right]\\
  =&e^{i(\omega'-\omega)t}(\omega'+\omega)(2\pi)^{n-1}
  \delta^{n-1}\left(\mathbf{k}-\mathbf{k'}\right)
\end{align}


we find that \(\left(f_{\mathbf{k}},f_{\mathbf{k}'}\right)=0\) for \(\mathbf{k}\neq\mathbf{k'}\).
Furthermore, if we choose the normalization constant \(A\) in eq \ref{112} as
\(\frac{1}{\sqrt{2\omega(2\pi)^{n-1}}}\), we find the mode function
\begin{equation}
  f_{\mathbf{k}}(x^{\mu})=\frac{e^{-ik_{\mu}x^{\mu}}}{\sqrt{2\omega(2\pi)^{n-1}}}
\end{equation}
obey
\begin{equation}
  \left(f_{\mathbf{k}},f_{\mathbf{k}'}\right)=\delta^{(n-1)}(\mathbf{k}-\mathbf{k}').
\end{equation}

Given our dispersion relation, \(\mathbf{k}\) only determines the absolute value of frequency.
However, we can require that all mode functions have positive frequency and still give a complete set of mode functions by
including complex conjugates \(f^*_{\mathbf{k}}(x^{\mu})\).

The positive frequency mode is defined as \[\frac{\p}{\p t}f_{\mathbf{k}}=-i\omega f_{\mathbf{k}}.\]
And the mode with negative frequency is \[\frac{\p}{\p t}f^*_{\mathbf{k}}=i\omega f^*_{\mathbf{k}}.\]

The negative frequency modes are orthogonal to the positive frequency modes:
\begin{equation}
  \left(f_{\mathbf{k}},f^*_{\mathbf{k}'}\right)=0.
\end{equation}
And they are orthonormal with each other with a negative norm:
\begin{equation}
  \left(f^*_{\mathbf{k}},f^*_{\mathbf{k}'}\right)=-\delta^{(n-1)}(\mathbf{k}-\mathbf{k'})
\end{equation}
Hence, modes \(f_{\mathbf{k}}\) and \(f^*_{\mathbf{k}}\) form a complete set, which any possible solution
of Klein-Gordon equation can be expressed in terms of them.


\subsection{Canonical Quantization}

The system could be quantized in the canonical quantization scheme by treating the field \(\phi\) as an operator \(\hat{\phi}\),
then impose the canonical commutation relations on equal-time hypersurface:
\begin{definition}[Canonical commutation relation]\label{121}
  \begin{align*}
    \bigl[\hat{\phi}(t,\mathbf{x}),\hat{\phi}(t,\mathbf{x'})\bigr] & =0                                       \\
    \bigl[\hat{\pi}(t,\mathbf{x}),\hat{\pi}(t,\mathbf{x'})\bigr]   & =0                                       \\
    \bigl[\hat{\phi}(t,\mathbf{x}),\hat{\pi}(t,\mathbf{x'})\bigr]  & =i\delta^{(n-1)}(\mathbf{x}-\mathbf{x'})
  \end{align*}
\end{definition}
The first two commutation relations come from the causality requirement, as those operators have spacelike separation.
The delta function implies that field and momentum operators commute everywhere except the spacetime point they intersect.
Just like the classical solution of the Klein-Gordon equation can be expanded in terms of mode,
the field operator \(\hat{\phi}\) also can be expanded in term mode function and have coefficients \(\hat{a}_{\mathbf{k}}\) and
\(\hat{a}^{\dagger}_{\mathbf{k}}\) respectively as shown below:
\begin{cthm}[Mode expansion]
  \[\phi(t,\mathbf{x})=\int\id^{n-1}k\left[\hat{a}_{\mathbf{k}}f_{\mathbf{k}}(t,\mathbf{x})+
    \hat{a}^{\dagger}_{\mathbf{k}}f^*_{\mathbf{k}}(t,\mathbf{x})\right]\]
\end{cthm}
By using the commutation relation defined in \ref{121}, we can obtain the commutation relation of operator \(\hat{a}_{\mathbf{k}}\) and
\(\hat{a}^{\dagger}_{\mathbf{k}}\):


\begin{align}
  \bigl[\hat{a}_{\mathbf{k}},\hat{a}_{\mathbf{k'}}\bigr]                     & =0                                      \\
  \bigl[\hat{a}^{\dagger}_{\mathbf{k}},\hat{a}^{\dagger}_{\mathbf{k'}}\bigr] & =0                                      \\
  \bigl[\hat{a}_{\mathbf{k}},\hat{a}^{\dagger}_{\mathbf{k'}}\bigr]           & =\delta^{(n-1)}(\mathbf{k}-\mathbf{k'})
\end{align}


Analog to harmonic oscillators, the operator \(\hat{a}_{\mathbf{k}}\) and
\(\hat{a}^{\dagger}_{\mathbf{k}}\) are annihilation and creation operator respectively.
The only difference is that we now have an infinite set of annihilation and creation operators corresponding to each spatial wave vector \(\mathbf{k}\).
%We can use annihilation and creation operators to define a basis for Hilbert space where the basis state is an eigenstate of number operator \(\)

\begin{remark}
  The quantization process described above is sometimes referred to as \tbf{second quantization}. Historically, this name
  comes from the fact that we first treat the mode as discrete and then have an integer number of excitation of each mode.
  However, the name ``second quantization'' can be misleading because the discrete mode is a classical phenomenon.
  We quantized the field exactly once.
\end{remark}


There is a single state \(\ket{0}\) that would be anihilated by all \(\hat{a}_{\mathbf{k}}\), called \tbf{vacuum}.
\begin{definition}[Vacuum]
  \[\forall\; \mathbf{k},\; \hat{a}_{\mathbf{k}}\ket{0}=0.\]
\end{definition}

A state with \(n\) particles with identical momentum \(\mathbf{k}\) can be constructed by repeat acting \(\hat{a}^{\dagger}_{\mathbf{k}}\)
on the vacuum:
\begin{equation}
  \ket{n_{\mathbf{k}}}=\frac{1}{\sqrt{n_{\mathbf{k}}!}}\left(\hat{a}^{\dagger}_{\mathbf{k}}\right)^{n}\ket{0}
\end{equation}

Similarly, we can construct a state with \(n_{i}\) particle for momentum \(\mathbf{k}_i\):
\begin{equation}
  \ket{n_1,n_2,\cdots,n_j}=\frac{1}{\sqrt{n_1!n_2!\cdots n_j!}}\left(\hat{a}^{\dagger}_{\mathbf{k}_1}\right)^{n_1}
  \left(\hat{a}^{\dagger}_{\mathbf{k}_2}\right)^{n_2}\cdots \left(\hat{a}^{\dagger}_{\mathbf{k}_j}\right)^{n_j}\ket{0}
\end{equation}

We can create or annihilate particles with certain momentum:
\begin{example}
  \begin{align*}
    \hat{a}_{\mathbf{k}_i}\ket{n_1,n_2,\cdots,n_i,\cdots,n_j}=           & \sqrt{n_i}\ket{n_1,n_2,\cdots,n_i-1,\cdots,n_j}   \\
    \hat{a}^{\dagger}_{\mathbf{k}_i}\ket{n_1,n_2,\cdots,n_i,\cdots,n_j}= & \sqrt{n_i+1}\ket{n_1,n_2,\cdots,n_i+1,\cdots,n_j} \\
  \end{align*}
\end{example}

Furthermore, we can define \tbf{number operator}:
\begin{definition}[Number operator]
  \[\hat{n}_{\mathbf{k}}=\hat{a}^\dagger_{\mathbf{k}}\hat{a}_{\mathbf{k}}\]
\end{definition}

Which obeys:
\begin{equation}
  \hat{n}_{\mathbf{k}_i}\ket{n_1,n_2,\cdots,n_i,\cdots,n_j}=    n_i\ket{n_1,n_2,\cdots,n_i,\cdots,n_j}
\end{equation}

The eigenstates of the number operator form a basis span Hilbert space, known as \tbf{Fock basis}.
The space span by this basis is called \tbf{Fock space}.

To illustrate the essential difference between flat and curved spacetime, we will examine how the Fock basis behaves under Lorentz transformation.
Consider a boost along \(x\) direction; the time derivative in the boosted frame is:
\begin{align}
  \p_{t'}f_{\mathbf{k}} & =\frac{\p x^{\mu}}{\p t'}\p_{\mu}f_{\mathbf{k}}                                   \\
                        & =\gamma(-i\omega)f_{\mathbf{k}}+\gamma\mathbf{v}\cdot (i\mathbf{k})f_{\mathbf{k}} \\
                        & =-i\omega'f_{\mathbf{k}}
\end{align}
where \[\omega'=\gamma\omega-\gamma\mathbf{v}\cdot \mathbf{k}\].
Hence, a state describing a collection of particles in the boosted frame would describe the identical particles with boosted momentum.
Therefore, the number operator in the boosted frame is identical to the original number operator; then, the vacuum will also be the same.
This particular result is due to the existence of a timelike Killing vector in flat spacetime, a direct consequence of Poincare symmetry.
With Poincare symmetry, our original choice of the inertial frame would be irrelevant.
Therefore, all inertial observers agree with a unique, single vacuum state.

\subsection{Vacuum Energy}
The Hamiltonian operator can be obtained from the classical theory of field in the same manner.
Recall the Hamiltonian of Klein-Gordon field is:
\begin{equation}
  H=\int\id^3x\;\mathcal{H}=\int\id^3x\left[\frac{1}{2}\pi^2+\frac{1}{2}(\nabla\phi)^2+\frac{1}{2}m^2\phi^2\right]
\end{equation}
By substituting the mode expansion of \(\hat{\phi}\), we obtained the expression of Hamiltonian of quantized K-G field:
\begin{align}
  \hat{H}=\frac{1}{2}\int \id^{n-1}k\left[\hat{a}^{\dagger}_{\mathbf{k}}\hat{a}_{\mathbf{k}}+\hat{a}_{\mathbf{k}}\hat{a}^{\dagger}_{\mathbf{k}}\right]\omega
\end{align}
Use the commutation relation of \(\hat{a}_{\mathbf{k}}\) and \(\hat{a}^{\dagger}_{\mathbf{k}}\), we can further simplify the Hamiltonian operator:
\begin{align}\label{134}
  \hat{H} & =\int \id^{n-1}k\left[\hat{a}^{\dagger}_{\mathbf{k}}\hat{a}_{\mathbf{k}}+\frac{1}{2}\delta^{(n-1)}(0)\right]\omega \\
          & =\int \id^{n-1}k\left[\hat{a}^{\dagger}_{\mathbf{k}}\hat{a}_{\mathbf{k}}\omega\right]+
  \int \id^{n-1}k\left[\frac{1}{2}\delta^{(n-1)}(0)\omega\right]
\end{align}
The problem has arisen: if we calculate the expectation value of Hamiltonian in the vacuum state, one would expect to get 0,
however, we get infinite.
The vacuum has infinite energy!
The first reason we see infinite vacuum energy is that we are integral over all space. This is reasonable, analog
to harmonic oscillator zero point energy,
if we sum over infinite many ground state harmonic oscillators, we are expecting infinite energy.
The divergences caused by infinitely large space are often referred to as \tbf{infrared divergences}.
We can eliminate this kind of infinite through \tbf{regularization}.
Let confine the field in a box with length \(L\) by imposing periodic boundary conditions, and rewrite the second term in \ref{134} as:
\begin{equation}\label{135}
  \int \id^{n-1}k\left[\frac{1}{2}\delta^{(n-1)}(0)\omega\right]\rightarrow\frac{1}{2}\left[\frac{L}{2\pi}\right]^{n-1}\sum_{\mathbf{k}}\omega
\end{equation}
We have used the Fourier transform of \(\delta\) function.

However, after we restrict the vacuum in a finite region, the expression in \ref{135} is still divergent.
Since the value of \(\omega=\sqrt{|\mathbf{k}|^2+m^2}\) can be arbitrarily large.
This infinite arises because we assumed quantum field theory is valid for arbitrarily high frequency/energy
which corresponds to arbitrarily short distance. We expect to see new physics at that energy scale!
The divergences caused by infinitely high frequency are often referred to as \tbf{ultraviolet divergences}.
We can eliminate this kind of infinite through \tbf{renormalization}.
The simplified idea is just substrating off infinite from our expression.
This is valid because what we can measure in the experiment is the energy difference,
we can simply rescale the zero point of energy and not affect the observable.




\subsection{Green Function}





\newpage

\section{QFT in Curved Spacetime}
In the Minkowski spacetime, we have a ``privileged'' set of basis to construct our mode function; however, in general spacetime,
due to the lack of Poincare symmetry, we do not have a set of mode functions that are preferred.


\subsection{Construct QFT in Curved Spacetime}
Let us start with the Lagrangian density of a scalar field:
\[\mathcal{L} = \frac{1}{2}(\eta^{\mu\nu} \phi_{,\mu} \phi_{,\nu}- m^2 \phi^2)\]
To obtain the curved spacetime expression,
simply replace the Minkowski metric by \(g^{\mu\nu}\) and replace the partial derivative with the covariant derivative.
To involve the coupling to curved spacetime background, we also introduce a coupling term
\begin{equation}
  \mathcal{L} = \frac{1}{2}\sqrt{g}(g^{\mu\nu} \phi_{;\mu} \phi_{;\nu}- m^2 \phi^2-\xi R \phi^2)
\end{equation}
Where the \(R\) is the Ricci scalar and \(\xi\) is the coupling constant.\par
There are two favorite choices of coupling constant:
\textbf{minimal coupling} which simply set \(\xi=0\) which turn off the coupling,
while \textbf{conformal coupling} sets \[\xi = \frac{(n-2)}{4(n-1)}\]\\
which makes the theory invariant under conformal transformation.


\subsection{Vacuum Ambiguity}


\subsection{Particle Detector and Unruh Effect}
To see what is the meaning of particles in curved spacetime, consider an idealized particle detector,
consisting of a point particle with internal energy level \(E\),
coupled via a monopole interaction with a scalar field \(\phi\).
Suppose the particle detector moves along the worldline described by parameterized function \(x^\mu(\tau)\),
where \(\tau\) is the detector's proper time.
\par The detector-field interaction is described by the interaction Lagrangian:
\begin{equation}
  L=\xi \hat{m}(\tau)\phi\left[x(\tau)\right]
\end{equation}
Where the \(\xi\) is the coupling constant and \(\hat{m}\) is the monopole moment of the detector.
Suppose the field \(\phi\) is in the Minkowski vacuum \(\ket{0_{M}}\), defined by \(a_{\mathbf{p}}\ket{0}=0\).
For the detector moves in a general trajectory,
it will not remain in the ground state \(E_0\) but will transition to an excited state \(E_n\) for \(n>0\),
while the field will transition to an excited state \(\ket{\psi}\). \par
If the coupling constant \(\xi\) is sufficiently small, the amplitude of transition could be given by perturbation theory:
\begin{equation}\label{TA1}
  i\xi \bra{E,\psi}\int_{-\infty}^{\infty}d\tau\;\hat{m}(\tau)\phi\left[x(\tau)\right]\ket{0_{M},E_{0}}
\end{equation}
In the Heisenberg picture, the time evolution of the monopole moment operator is given by
\[\hat{m}(\tau)= e^{iH_0\tau}\hat{m}(0)e^{-iH_0\tau}.\]
Factorized the equation \ref{TA1},



------------@TODO

For an inertial detector in n-dimensional Minkowski spacetime, only the last term of xxx contributes to the detector response function:
\begin{equation}
  \frac{F(E)}{T}=(2\pi)^{1-n}\int^{\infty}_{-\infty}\id (\Delta \tau)\;e^{-iE\Delta\tau}
  \int\frac{\id^{n-1}k}{2\omega}e^{i(\omega-\mathbf{k}\cdot\mathbf{v})\frac{\Delta\tau}{\sqrt{1-v^2}}}n_{\mathbf{k}}
\end{equation}
Where \(T\) is the duration of the detector switched on. If \(\mathbf{v}=0\) we can further perform the integration:
\begin{align}
  \frac{F(E)}{T} =& (2\pi)^{1-n}\int^{\infty}_{-\infty}\id (\Delta \tau)\;e^{-i\Delta\tau E}
  \int\frac{\id^{n-1}k}{2\omega}e^{i\Delta\tau\frac{(\omega-\mathbf{k}\cdot\mathbf{v})}{\sqrt{1-v^2}}}n_{\mathbf{k}}       \\
  =&(2\pi)^{1-n}\int\frac{\id^{n-1}k}{2\omega}\int^{\infty}_{-\infty}\id (\Delta \tau)\;e^{-i\Delta\tau E}
  e^{i\Delta\tau\frac{(\omega-\mathbf{k}\cdot\mathbf{v})}{\sqrt{1-v^2}}}n_{\mathbf{k}}                                     \\
  =&(2\pi)^{1-n}\int\frac{\id^{n-1}k}{2\omega}\int^{\infty}_{-\infty}\id (\Delta \tau)\;
  e^{i\Delta\tau\frac{(\omega-\mathbf{k}\cdot\mathbf{v})}{\sqrt{1-v^2}}-i\Delta\tau E}n_{\mathbf{k}}                       \\
  =&(2\pi)^{1-n}\int\frac{\id^{n-1}k}{2\omega}\int^{\infty}_{-\infty}\id (\Delta \tau)\;
  e^{i\Delta\tau\left[\frac{(\omega-\mathbf{k}\cdot\mathbf{v})}{\sqrt{1-v^2}}-E\right]}n_{\mathbf{k}}                      \\
  =&(2\pi)^{1-n}(2\pi)\int\frac{\id^{n-1}k}{2\omega}\;
  \delta\left(\frac{(\omega-\mathbf{k}\cdot\mathbf{v})}{\sqrt{1-v^2}}-E\right)n_{\mathbf{k}}                               \\
  =&(2\pi)^{1-n}(2\pi)\frac{2\pi^{\frac{n-1}{2}}}{\Gamma(\frac{n-1}{2})}\int\frac{k^{n-2}\id k}{2\omega}\;
  \delta\left(\omega-E\right)n_{\mathbf{k}}                                                                                \\
  =&\frac{2^{2-n}\pi^{\frac{3-n}{2}}}{\Gamma(\frac{n-1}{2})}\int\frac{k^{n-2}\id k}{\omega}\;
  \delta\left(\omega-E\right)n_{\mathbf{k}}                                                                                \\
  =&\frac{2^{2-n}\pi^{\frac{3-n}{2}}}{\Gamma(\frac{n-1}{2})}\int\frac{k^{n-2}\id k}{\sqrt{k^2+m^2}}\;
  \delta\left(\sqrt{k^2+m^2}-E\right)n_{\sqrt{E^2-m^2}}                                                                    \\
  =&\frac{2^{2-n}\pi^{\frac{3-n}{2}}}{\Gamma(\frac{n-1}{2})}\left(E^2-m^2\right)^{\frac{n-3}{2}}
  \theta\left(E-m\right)n_{\sqrt{E^2-m^2}}
\end{align}




\section{Quantum Phenomena In Curved Spacetime}


\subsection{Cosmological Particle Creation}



\subsection{Moving Mirror}


\subsection{Hawking Radiation}

%\section{Renormalization}

\newpage
\appendix
\addcontentsline{toc}{section}{Appendix~}





\section{Classical Field Theory}
Classical field theory is something students are supposed to know for QFT but is never formally taught, so we give a brief overview here.
Field theory is very similar to Lagrangian mechanics; instead of the usual system,
we now have a spacetime-dependent \textbf{fields} \(\Phi^i(x^\mu)\) (the \(i\) here is not tensor index),
and the action becomes a \textbf{functional} of these fields. A functional is a ``function of function'', which map a function to a number,
note that the functional is not simply a composition function, which maps number to number. \par
In field theory, the Lagrangian is usually expressed as an integral over \textbf{Lagrange density},
which are a function of the fields and their derivative.
\begin{equation}
  L=\int d^3x\;\mathcal{L}(\Phi^i,\p_{\mu}\Phi^i)
\end{equation}
Then the action
\begin{equation}
  S=\int dt\;L = \int d^4x\;\mathcal{L}(\Phi^i,\p_{\mu}\Phi^i)
\end{equation}
By varying the action, just like what we did in classical mechanics, we can obtain the equation of motion of field:

\begin{align}
  \delta S & = \int d^4x\left[\frac{\p\mathcal{L}}{\p \Phi^i}\delta\Phi^i+
  \frac{\p\mathcal{L}}{\p(\p_\mu \Phi^i)}\p_\mu (\delta\Phi^i)\right]      \\
           & =\int d^4x\left[\frac{\p\mathcal{L}}{\p \Phi^i}-\p_\mu
    \frac{\p\mathcal{L}}{\p(\p_\mu \Phi^i)}\right]\delta\Phi^i
\end{align}

We have obtained the field version of \textbf{Euler-Lagrange equation}
\begin{definition}[Euler-Lagrange equation]
  \[\frac{\p\mathcal{L}}{\p \Phi^i}-\p_\mu
    \frac{\p\mathcal{L}}{\p(\p_\mu \Phi^i)}=0\]
\end{definition}
Let's consider a simple example, the real scalar field \[\phi:\ x^{\mu} \rightarrow \mathbb{R}\].
The contribution of action are

\begin{enumerate}
  \item Kinetic term: \(\frac{1}{2}\dot{\phi}^2\)
  \item Gradient term: \(\frac{1}{2}(\nabla\phi)^2\)
  \item potential term: \(V(\phi)\)
\end{enumerate}
We could combine them into a Lorentz-invariant Lagrange density:
\begin{equation}
  \mathcal{L}= \frac{1}{2}\eta^{\mu\nu}(\p_{\mu}\phi)(\p_{\nu}\phi)-V(\phi)
\end{equation}
Apply the Euler-Lagrange equation; we then get the equation of motion:
\begin{cthm}
  \begin{align*}                                & \eta^{\mu\nu}\p_{\mu}\p_{\nu}\phi+\frac{\mathrm{d}V(\phi)}{\mathrm{d}\phi} \\
                                          =\  & \square\phi+\frac{\mathrm{d}V(\phi)}{\mathrm{d}\phi}                       \\
                                          =\  & 0
  \end{align*}
  Where \(\square\) is \textbf{d'Alembertian}.
\end{cthm}
\subsection{Klein-Gordon Field}
If the scalar field is massive, the potential \(V(\phi)=\frac{1}{2}m^2\phi^2\),
we obtained the \textbf{Klein-Gordon equation}:
\begin{definition}[Klein-Gordon equation]
  \[(\square +m^2)\phi = 0\]
\end{definition}
You will see this equation again and again in quantum field theory.
\begin{remark}
  The Klein-Gordon field could be analog to infinite many coupled infinitesimal harmonic oscillators;
  each ``mass'' is affected by neighboring springs and has its kinetic energy.
\end{remark}



\subsection{Hamiltonian Field Theory}
The Lagrangian field theory is naturally Lorentz invariant. However, we also need Hamiltonian formalism for the field theory
because it is easier to transition from quantum mechanics.
Let's first consider the classical definition of Hamiltonian, \[H=\sum p\dot{q}-L\].
From the definition of Lagrangian, we can derive the conjugate momentum as follow:
\begin{cthm}
  \begin{align*}
    p(x^i)\equiv & \;\frac{\p L}{\p \dot{\phi}(x^i)}
    =\frac{\p}{\p \dot{\phi}(x^i)}\int \mathrm{d}^3x' \mathcal{L} \\
    \propto      & \;\pi(x^i) \mathrm{d}^3x
  \end{align*}
  \bigskip
  Where \[\pi \equiv \frac{\p \mathcal{L}}{\p \dot{\phi}}\]
  is called \textbf{momentum density}
\end{cthm}
Therefore, the Hamiltonian for field theory is:
\begin{definition}[Hamiltonian]
  \begin{align*}
    H= & \sum p(x^i)\phi(x^i)-L                             \\
    =  & \int \id^3x\left[\pi \dot{\phi}-\mathcal{L}\right] \\
    =  & \int \id^3x\;\mathcal{H}
  \end{align*}
  Where \(\mathcal{H}\) is called \tbf{Hamiltonian density}.
\end{definition}

\begin{example}
  As a simple example, the Hamiltonian for the K-G field is:
  \[H=\int\id^3x\;\mathcal{H}=\int\id^3x\left[\frac{1}{2}\pi^2+\frac{1}{2}(\nabla\phi)^2+\frac{1}{2}m^2\phi^2\right]\]
\end{example}

\subsection{Noether's Theorem}
A Lagrangian may be invariant under some special type of transformation.
For example, a Lagrangian for a complex scalar field
\[\mathcal{L}= |\p_\mu \phi|^2 -m^2|\phi|^2\]
is invariant under transformation \(\phi \rightarrow e^{i\alpha}\phi\).
We call this transformation a \textbf{symmetry} of the Lagrangian.
When the parameter of transformation (\(\alpha\) for this case) can be taken infinitesimal,
we say this symmetry is \textbf{continuous}.
We can explicit see that \(\frac{\delta \mathcal{L}}{\delta \alpha}=0\).
\par By applying the Euler-Lagrange equation, we can deduce that \(\p_\mu J_{\mu}=0\),
where \(J_{\mu}\) is defined as follow:
\begin{definition}
  \[J_{\mu}= \sum_n\frac{\p \mathcal{L}}{\p(\p_\mu \phi_n)}\frac{\delta\phi_n}{\delta \alpha}\]\\
  \(J_{\mu}\) is known as \textbf{Noether current} or \textbf{conserved current}
\end{definition}

\begin{example}
  For the above complex scalar field, we can calculate the conserved current:
  \bigskip
  \begin{align*}
    J_{\mu}= & \sum_n\frac{\p \mathcal{L}}{\p(\p_\mu \phi_n)}\frac{\delta\phi_n}{\delta \alpha} \\
    =        & \frac{\p \mathcal{L}}{\p(\p_\mu \phi)}\frac{\delta\phi}{\delta \alpha}
    +\frac{\p \mathcal{L}}{\p(\p_\mu \phi^*)}\frac{\delta\phi^*}{\delta \alpha}                 \\
    =        & -i\left[\phi\p_\mu\phi^*-\phi^* \p_\mu\phi\right]
  \end{align*}
  We can check that \(\p_\mu J_{\mu}=0\) explicitly:
  \begin{align*}
    \p_\mu J_{\mu}= & -i\left[\phi\square\phi^*-\phi^* \square\phi\right] \\
    =               & 0
  \end{align*}
  Where we applied the equation of motion in the last step.
\end{example}
We call \(J_\mu\) conserved current because we can find conserved quantity by integral over its 0-component:
\begin{definition}[Conserved charge]
  \[Q=\int d^3x\;J_0\]
  Where the \(Q\) is called \textbf{conserved charge} or \textbf{Noether's charge}
\end{definition}
We say \(Q\) is conserved because
\begin{equation}
  \p_tQ=\int d^3x \; \p_t J_0 = \int d^3x\;\vec{\nabla}\cdot \vec{J} = 0.
\end{equation}
The above argument is called \textbf{Noether's theorem}.
\begin{theorem}[Noether's theorem]
  If a Lagrangian has a continuous symmetry,
  then exists a current associated with that symmetry that is
  conserved when the equations of motion are satisfied.
\end{theorem}

\subsection{Energy-momentum Tensor}
The physics at spacetime point \(x\) should have same form at spacetime point \(y\),
this argument arises a symmetry called \textbf{specetime translational symmetry}.
By Noether's theorem, we can find the four conserved currents for this symmetry for an infinitesimal
spacetime translation \(\xi^\mu\):
\begin{definition}[Energy-momentum tensor]
  \[T_{\mu\nu}=\sum_n \frac{\p \mathcal{L}}{\p(\p_{\mu}\phi_n)}\p_\nu\phi_n-g_{\mu\nu}\mathcal{L}\]

  \(T_{\mu\nu}\) is called \textbf{energy-momentum tensor}.
\end{definition}
Four conserved currents correspond to values of \(\nu\), and four conserved charges are energy and momentum.
We can see that the energy density is the 00-component of \(T_{\mu\nu}\).




\newpage
%\section{Driven Harmonic Oscillator}
\newpage



\end{document}